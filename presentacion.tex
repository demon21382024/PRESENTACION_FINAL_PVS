\documentclass[aspectratio=169]{beamer}

% --- TEMA Y COLORES ---
\usetheme{Madrid}
\usecolortheme{dolphin} % Tema azul profesional

% --- PAQUETES ---
\usepackage[spanish]{babel} % Idioma español
\usepackage[utf8]{inputenc}
\usepackage{booktabs}       % Tablas estéticas
\usepackage{graphicx}       % Imágenes

% --- METADATOS ---
\title[Informe Técnico Final]{Informe Técnico Final \\ Implementación Pipeline CI/CD}
\subtitle{Proyecto Bookztron}
\author{Harold Alexis Victor Canto Vidal}
\institute{Curso: Prueba y Verificación de Software}
\date{28 de Noviembre, 2025}

\begin{document}

% --- SLIDE 1: TÍTULO ---
\begin{frame}
    \titlepage
\end{frame}

% --- SLIDE 2: RESUMEN EJECUTIVO ---
\begin{frame}{Resumen Ejecutivo}
    \begin{block}{Objetivo Principal}
        Diseñar e implementar un flujo de trabajo automático (Pipeline CI/CD) en \textbf{GitLab} para validación continua.
    \end{block}

    \vspace{0.5cm}

    \begin{block}{Logros Clave}
        \begin{itemize}
            \item \textbf{Funcionalidad:} Pruebas unitarias y de interfaz integradas.
            \item \textbf{Seguridad:} Escaneos de vulnerabilidades automatizados.
            \item \textbf{Rendimiento:} Pruebas de carga y estrés en la infraestructura.
        \end{itemize}
    \end{block}
\end{frame}

% --- SLIDE 3: ARQUITECTURA ---
\begin{frame}{Arquitectura e Infraestructura}
    \begin{columns}
        \column{0.5\textwidth}
            \textbf{Diseño del Pipeline:}
            \begin{itemize}
                \item 10 etapas secuenciales y paralelas (Build, Test, Security, Analyze, Quality Gate, Performance).
            \end{itemize}
            
            \vspace{0.5cm}
            
            \textbf{Runners Locales (Híbrido):}
            \begin{itemize}
                \item \textbf{Laptop MSI (\#50601279):} Ejecuciones ágiles.
                \item \textbf{PC Escritorio (\#50583101):} Carga pesada.
            \end{itemize}

        \column{0.5\textwidth}
            % Placeholder para la imagen de arquitectura
            \begin{figure}
                \centering
                \framebox{\parbox[c][5cm][c]{0.9\textwidth}{\centering
                    \textbf{Diagrama de Arquitectura} \\
                    \small (Insertar imagen aquí)
                }}
            \end{figure}
    \end{columns}
\end{frame}

% --- SLIDE 4: MATRIZ DE HERRAMIENTAS ---
\begin{frame}{Matriz de Herramientas y Resultados}
    \begin{table}
        \centering
        \small
        \begin{tabular}{lllc}
            \toprule
            \textbf{Prueba} & \textbf{Herramienta} & \textbf{Resultado Clave} & \textbf{Estado} \\
            \midrule
            Estilo & ESLint & Cumple reglas (Google/Airbnb) & \textcolor{green!60!black}{\textbf{Bien}} \\
            Unitarias & Jest & 40.5\% Cobertura & \textcolor{orange}{\textbf{Mejorar}} \\
            API & Newman & Respuestas rápidas ($\sim$200ms) & \textcolor{green!60!black}{\textbf{Bien}} \\
            UI & Selenium & Fallos de diseño (CSS) & \textcolor{orange}{\textbf{Hallazgos}} \\
            Rendimiento & JMeter & Pasarela lenta (5-8s) & \textcolor{red}{\textbf{Crítico}} \\
            Seguridad & OWASP ZAP & Faltan headers de seguridad & \textcolor{orange}{\textbf{Alerta}} \\
            \bottomrule
        \end{tabular}
    \end{table}
\end{frame}

% --- SLIDE 5: UNITARIAS Y API ---
\begin{frame}{Detalles: Unitarias y API}
    \begin{itemize}
        \item \textbf{Pruebas Unitarias (Jest):}
        \begin{itemize}
            \item Resultado actual: \textbf{40.5\% de Cobertura}.
            \item Análisis: El código pasa las pruebas, pero la cobertura es insuficiente para garantizar estabilidad total.
        \end{itemize}
        
        \vspace{1cm}
        
        \item \textbf{Pruebas de API (Newman):}
        \begin{itemize}
            \item Resultado: Backend simulado responde correctamente.
            \item Status: \textbf{200 OK} en todos los endpoints críticos.
        \end{itemize}
    \end{itemize}
\end{frame}

% --- SLIDE 6: VISUALES Y RENDIMIENTO ---
\begin{frame}{Detalles: Visuales y Rendimiento}
    \begin{columns}
        \column{0.6\textwidth}
            \begin{block}{Pruebas Visuales (Selenium)}
                \begin{itemize}
                    \item Hallazgo: "Parpadeo" inicial en menú de navegación.
                    \item Impacto: Afecta la percepción de calidad del usuario.
                \end{itemize}
            \end{block}
            
            \vspace{0.5cm}
            
            \begin{alertblock}{Rendimiento (JMeter) - Crítico}
                \begin{itemize}
                    \item Cuello de Botella: Flujo de pagos.
                    \item Tiempos: \textbf{5 a 8 segundos} (Inaceptable para producción).
                \end{itemize}
            \end{alertblock}
            
        \column{0.4\textwidth}
            % Placeholder para gráfico JMeter
            \begin{figure}
                \centering
                \framebox{\parbox[c][4cm][c]{0.9\textwidth}{\centering
                    \small Gráfico JMeter
                }}
            \end{figure}
    \end{columns}
\end{frame}

% --- SLIDE 7: SEGURIDAD Y CALIDAD ---
\begin{frame}{Seguridad y Calidad de Código}
    \begin{itemize}
        \item \textbf{Seguridad (OWASP ZAP):}
        \begin{itemize}
            \item Vulnerabilidad: Ausencia de cabeceras HTTP de seguridad.
            \item Vulnerabilidad: Falta de \textbf{CAPTCHA} en login (Riesgo bots).
        \end{itemize}
        
        \vspace{0.8cm}
        
        \item \textbf{Calidad (SonarQube):}
        \begin{itemize}
            \item Calificación General: \textbf{A}.
            \item Deuda Técnica: 10 errores menores y 11 Security Hotspots a revisar.
            \item Quality Gate: \textbf{Aprobado}.
        \end{itemize}
    \end{itemize}
\end{frame}

% --- SLIDE 8: GESTIÓN DE ERRORES ---
\begin{frame}{Gestión de Errores - Hallazgos Críticos}
    \textbf{Registro en MantisBT:}
    
    \vspace{0.5cm}
    
    \begin{enumerate}
        \item \textbf{\textcolor{red}{[CRÍTICA] ID 0000015:}} Pagos Lentos (5-8s).
        \item \textbf{\textcolor{orange}{[ALTA] ID 0000013:}} Login sin CAPTCHA (Riesgo Fuerza Bruta).
        \item \textbf{\textcolor{orange}{[ALTA] ID 0000016:}} Inconsistencia en Carrito (No sincroniza entre pestañas).
        \item \textbf{\textcolor{green!60!black}{[MEDIA] ID 0000014:}} Persistencia de preferencias tras Logout.
    \end{enumerate}
\end{frame}

% --- SLIDE 9: ANÁLISIS CRÍTICO ---
\begin{frame}{Análisis Crítico del Proyecto}
    \begin{block}{Infraestructura}
        \begin{itemize}
            \item \textbf{Pros:} Runners propios reducen costos y tiempos.
            \item \textbf{Contras:} Riesgos operativos (cortes de luz/red) en entorno local.
        \end{itemize}
    \end{block}
    
    \vspace{0.3cm}
    
    \begin{block}{Estrategia de Pruebas}
        Existe un desbalance: Alta dependencia de pruebas E2E y poca cobertura unitaria.
    \end{block}

    \vspace{0.3cm}

    \begin{block}{Seguridad}
        Integración tardía, faltando enfoque "Shift-Left" desde el diseño.
    \end{block}
\end{frame}

% --- SLIDE 10: CONCLUSIONES ---
\begin{frame}{Conclusiones y Futuro}
    \begin{columns}
        \column{0.5\textwidth}
            \textbf{Éxitos del Proyecto:}
            \begin{itemize}
                \item Automatización total (JS/Java/Python).
                \item Infraestructura híbrida funcional.
                \item Trazabilidad completa en MantisBT.
            \end{itemize}
        
        \column{0.5\textwidth}
            \textbf{Acciones Futuras:}
            \begin{itemize}
                \item Elevar cobertura Jest $>$ 80\%.
                \item Optimizar base de datos para pagos.
                \item Hardening de seguridad antes de producción.
            \end{itemize}
    \end{columns}
\end{frame}

% --- SLIDE 11: PREGUNTAS ---
\begin{frame}
    \centering
    \Huge \textcolor{structure}{¿Preguntas?}
    
    \vspace{1cm}
    
    \Large Gracias por su atención.
\end{frame}

\end{document}